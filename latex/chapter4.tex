\chapter{Problem Statement}\label{chap:chap4}

\section*{}

%% FIXME: Problem chapter
%O capítulo sobre o problema que devia aparecer neste ponto, precisa de conter:
% - descrição geral do problema (possivelmente repetindo um pouco, mas não inteiramente, do que foi dito no capítulo da introdução) e âmbito que pretendemos atacar
% - a hipótese central subjacente ao trabalho
% - as sub-hipóteses ou research questions mais concretas
% - qual a estratégia de investigação que será usada

Smart contracts power a decentralized world of automation and trust-less commitments. Companies, groups and individuals are able to automate tasks and contracts but as far as the ecosystem is, smart contracts are still very much limited to the information available in the blockchain. Therefore, connecting with the outside world requires a trusted authority to input in the blockchain the required information upon request from the smart contract. This trusted authority is generally called an oracle.

As explained before, the deterministic nature of blockchain does not allow smart contracts to directly query a data-feed for information. In this context, oracles help connecting smart contracts to the world outside of the blockchain. The problem here is to trust in the oracle service to not behave maliciously and undermine the trust provided by the blockchain consensus mechanisms. Blockchain technology can be trusted to behave correctly even in byzantine environments, but the oracle service does not abide by the same mechanisms and therefore some mechanisms must be put in action to ensure the oracle's response credibility.

As seen in chapter~\ref{chap:sota}, current solutions to the oracle problem use complex mechanisms to achieve a certain desired level of trust. Some use complex trusted hardware others incentive based mechanisms or authenticity proofs, neither of these are simple and fully trusted approaches and add extra complexity from the developer side. Either he if he needs to implement it or if he has to analise how current oracle-as-a-service providers are using it.

The oracle problem is neither simple nor has a single solution, but its importance in powering greater applications for smart contracts is undeniable and its forces can be summarized in the following bullet points:

%% acredito que ao fazer x tenho y beneficios

\begin{itemize}
    \item \textbf{Smart contract empowerment} - Providing smart contracts with trustable information from outside of the blockchain is decisive to gain general adoption and practicality.
    \item \textbf{Blockchain interoperability} - The ability to get information from other blockchains if possible using an oracle that queries one blockchain and inputs the information on another.
    \item \textbf{Keeping trust standards} - As blockchain technology creates a trust-less environment, oracles should as well keep up with the level of trust in their functioning.
\end{itemize}

\section{Proposal}

With this thesis, I intend to lay the foundations for the development and architecture of trustable oracle systems that will power several smart contract use cases. The author believes that by describing, in a trust-guided manner, multiple patterns and examples where they are being applied or possible use cases not yet documented creates a guided model that should help future cases to have a systematic approach to which architecture will fit the best.

Furthermore, in chapter \ref{chap:chap6}, the author presents a possible implementation of a self-hosted oracle. The author believes that existing oracle services, and after analysing the used authenticity proofs, that the best way to have trust in an oracle is to deploy one instead of relying on a third-party. The described approach reduces operations costs, increases trust and empowers the contract with purpose built oracles.