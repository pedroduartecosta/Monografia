\chapter{Problem}\label{chap:chap4}

\section*{}

%% FIXME: Problem chapter
%O capítulo sobre o problema que devia aparecer neste ponto, precisa de conter:
% - descrição geral do problema (possivelmente repetindo um pouco, mas não inteiramente, do que foi dito no capítulo da introdução) e âmbito que pretendemos atacar
% - a hipótese central subjacente ao trabalho
% - as sub-hipóteses ou research questions mais concretas
% - qual a estratégia de investigação que será usada

\subsubsection{Context}
Smart contracts power a decentralized world of automation and trust-less commitments. Companies, groups and individuals are able to automate tasks and contracts but as far as the ecosystem is smart contracts are still very much limited to the information available in the blockchain. Therefore, connecting with the outside world requires a trusted authority to input in the blockchain the required information upon request from the smart contract. This trusted authority is generally called an oracle. 

As explained before, the deterministic nature of blockchain does not allow smart contracts to directly query a data-feed for information. In this context, oracles surge to empower business and smart contract capabilities, connecting smart contracts to the world outside of the blockchain. The problem here is to trust in the oracle service to not behave maliciously and undermine the trust provided by the blockchain consensus mechanisms. Blockchain technology can be trusted to behave correctly even in byzantine environments, but the oracle service does not abide by the same mechanisms and therefore must provide some kind of proof of honesty.

\subsubsection{Forces}
\begin{itemize}
\item \textbf{Smart contract empowerment} - Providing smart contracts with trustable information from outside of the blockchain is decisive to gain general adoption and practicality.
\item \textbf{Blockchain Interoperability} - The ability to get information from other blockchains if possible using an oracle that queries one blockchain and inputs the information on another.
\item \textbf{Keeping trust standards} - As blockchain technology creates a trust-less environment, oracles should as well keep up with the level of trust in their functioning.
\end{itemize}


\subsubsection{Solution}
A solution is to use a secure authenticated channel between any external data-source and blockchain applications, known as Oracle. Fee-based oracles allow smart contracts to pay for each request. In order to achieve a higher level of trust, for an extra fee, it can also provide proofs that the provided data as not been tampered by their oracle. The extent to which this proofs can be trusted will be further analysed.