\chapter{Problem Statement}\label{chap:chap4}

\section*{}

%% FIXME: Problem chapter
%O capítulo sobre o problema que devia aparecer neste ponto, precisa de conter:
% - descrição geral do problema (possivelmente repetindo um pouco, mas não inteiramente, do que foi dito no capítulo da introdução) e âmbito que pretendemos atacar
% - a hipótese central subjacente ao trabalho
% - as sub-hipóteses ou research questions mais concretas
% - qual a estratégia de investigação que será usada

Smart contracts power a decentralized world of automation and trust-less commitments. Companies, groups and individuals are able to automate tasks and contracts but in the current ecosystem, smart contracts are still very much limited to the information available in the blockchain. Therefore, connecting with the outside world requires a trusted authority to input in the blockchain the required information upon request from the smart contract. This trusted authority is generally called an oracle.

As explained before, the deterministic nature of blockchain does not allow smart contracts to directly query a data-feed for information. In this context, oracles help connecting smart contracts to the world outside of the blockchain. The problem here is to trust the oracle service to not behave maliciously and undermine the trust provided by the blockchain consensus mechanisms. Blockchain technology can be trusted to behave correctly even in byzantine environments, but the oracle service does not abide by the same rules and therefore some workings must be put into action to ensure the oracle's response credibility.

As seen in Chapter~\ref{chap:sota}, current solutions to the oracle problem use complex techniques to achieve a certain desired level of trust. Some use complicated trusted hardware others incentive-based mechanisms or authenticity proofs, but neither of these are simple and fully trusted approaches and they add extra complexity from the developer side.

The oracle problem is neither simple nor has a single solution, but its importance in powering greater applications for smart contracts  is undeniable. Their need arises from the following three factors.

%% acredito que ao fazer x tenho y beneficios

\begin{itemize}
    \item \textbf{Smart contract empowerment} - Providing smart contracts with trustable information from outside of the blockchain is decisive to gain general adoption and practicality.
    \item \textbf{Cost optimization} - Blockchain operations tend to be quite expensive, therefore, the oracle solution should introduce a lower overhead cost as possible.
    \item \textbf{Keeping trust standards} - As blockchain technology creates a trust-less environment, oracles should as well keep up with the level of trust in their functioning.
\end{itemize}

\section{Proposal}

With this thesis, the author intends to lay the foundations for the development and architecture of trustable oracle systems that will power several smart contract use cases.

The author believes that by describing, in a trust-guided manner, multiple patterns and examples where they are being applied or possible use cases not yet documented creates a guided model that helps future cases to have a systematic approach to which architecture will fit the best. The architecture and design of blockchain oracles is still very much unexplored territory, specially in terms of academia research but also in the industry, and therefore this thesis approaches it broadly and investigates possible approaches and their trade-offs so that later studies can be developed on the specifics of each architecture.

Furthermore, in Chapter \ref{chap:chap6}, the author presents a possible implementation of a self-hosted oracle. After analysing the state of the art in oracle development and the specifics of used authenticity proofs, the author believes that the best way to achieve trust in an oracle is to deploy one instead of relying on a third-party. The described approach, when in comparison to deployed solutions in the industry reduces operations costs, increases trust and empowers the contract with purpose built oracles. The author will demonstrate that deploying and oracle, can be more trivial than at first seems, and that trust in its operation is directly the trust in one's code and no more measures (authenticity proofs) are need. This measures usually add a considerable extra cost and constrain the developer.


\section{Desiderata}\label{Desiderata}
This section describes several forces that separately or in combination drive the design of oracle architectures and implementations. Defining this forces will help validate each architecture/implementation in what forces they are able to accomplish in their different scenarios.

\subsection{Fast time-to-market}
Not having to assemble a team or allocate resources into a developing a new product which will only serve as component of the main product being developed.

\subsection{Keeping trust standards}
The company focus is not the development and security of the  oracle  service and  may  not  have  enough  resources  to  keep  the  oracle  as  secure  and reliable as the underlying blockchain.

\subsection{Data-feed fault tolerance}
Ensuring  that  a  contract  can  follow  through  even  if  a  data provider is down by querying another provider.

\subsection{Data veracity}
Querying several data sources guarantees a higher trust on the veracity of the data by not allowing a single service to be the owner o the truth.

\subsection{Lower smart contract costs}
Checking authenticity proofs leads to higher contract deployment costs, as the proof can be long and computationally expensive. Striving to build simple oracle smart contracts will lead to reduced costs.

\subsection{Lower oracle complexity}
Dealing with authenticity proofs and implementing the necessary mechanisms verify them requires a higher and deeper knowledge of their implementation mechanisms and underlying cryptography.

\subsection{Oracle decentralization}
Connecting a smart contract to data through a single node, creates the problem that smart contracts intend to avoid, a single point of failure.  With a single oracle, a smart contract is only as reliable as that one oracle.

\subsection{Oracle ownership decentralization}
Having one party control the oracle network centralizes the power to manipulate all the contracts relying on the information provided by that network of oracles.


\section{Conclusions}
This project aims to pave the way for oracle and smart contract development. It does not try to come up with a new authenticity proof which adds extra complexity for the common smart contract developer, but instead guide the developer to a solution accordingly to the problem necessities. As well as, providing a simple but yet effective implementation of a self-hosted oracle so as to have a simple skeleton to which the developer can iterate upon and adapt to the specific smart contract needs.