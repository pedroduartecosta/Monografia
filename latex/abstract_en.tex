\chapter*{Abstract}


Technology always tends to move towards decentralization, the internet itself promised a decentralized world that would power each individual to connect, automate and thrive in a new digital world. However, the internet and it's information became more and more centralized and, when least expected and most desired, in the economical crisis of 2008, a new disruptive technology surged. Blockchain, powering Bitcoin, and setting a new standard for decentralized currency this technology guarantees no need for centralized authorities through the use of a byzantine fault-tolerant ledger, where information is recorded and verified by every node in the network, and immutable, once verified.

Just like natural evolution, out of need, this technology evolved to not only be able to store transactions, but also to perform computations with the same guarantees of decentralization and trust in the outcome. This computations are referred to as smart-contract. Capable of replacing existing traditional contracts but also, by being code, it can replace all ranges of application that ingenuity can dream of.

However, currently smart contracts lack an important feature, trivial to most applications, internet connectivity. Due to the deterministic nature of Blockchain, meaning that it gives certainties that if someone downloads and runs the whole chain it will always get the same outcome. Having said that, the internet is non-deterministic, it's information is ever changing and no API can be guaranteed to always retrieve the same result. Therefore smart-contracts cannot directly query an API, as doing so might return different answers at different times. But without internet connectivity smart-contracts will always have limited functionality.

Oracles solve the connectivity problem, by listening to events produced by smart-contracts, they can insert the needed information on the Blockchain to later be used by the contracts. But oracles do not abide by the same rules and guarantees given by Blockchain, so they must either be trusted or find ways to limit trust surface, and that is the main problem around oracles. To find how the oracle problem is being addresses a systematic literature review was performed and found little to no practical solutions. And therefore, the author also searched in forums on the web for projects from the industry. Existing oracle solutions rely on two main solutions. Authenticity proofs, which are cryptographic proofs that something actually happened or wisdom of the crowd solutions based on incentives, penalising bad behaviour in order to obtain true answers. The former, mostly employed by third-party providers is not fully transparent and always requires some level of trust, and the latter requires a large community and its bad behaviour can always go unchecked.

Bearing this in mind, this dissertation contribution is threefold. Firstly, it analyses and summarises the existing authenticity proofs and mechanisms for guaranteeing oracle trust. Allowing a smart-contract developer to be fully informed of the implications os using each proof and their limitations which are usually obfuscated by the providers. Secondly, it defines four possible architectures for oracle design and how each of them addresses different points of trust in the oracle, either in terms of oracle untampered relay of data as well as the veracity of the same data. Two of them from the perspective of using a third-party provider and the last two from a self-hosted perspective. This shall allow future oracle developers to think, before hand, of the limitations, compromises and possibilities of each way to developer a oracle. Finally, as the author worked on each architecture he found existing oracle solutions that would fit in the first two, regarding third-party providers, and later a new project focusing on development of the third architecture but not quite there yet. Having this in mind, the author decided to implement the fourth architecture, both because no existing solution was available and because doing so would satisfy the requirements of both the fourth and third architecture, as the former complements the latter. With this last contribution the author adds to the community a solution usually provided by third-parties and not transparent in its implementation. Also a solution that, by being simple can be easily tailored to satisfy different project needs.

In conclusion, this dissertation paves the way for oracle development and research summarising firstly in a broad sense existing solutions and later contributing with a systematic set of architectures and a solution that can be easily adopted by teams interested in deploying their own oracle to achieve higher standards of trust.


\chapter*{Resumo}



%O Resumo fornece ao leitor um sumário do conteúdo da dissertação.
%Deverá ser breve mas conter detalhe suficiente e, uma vez que é a porta
%de entrada para a dissertação, deverá dar ao leitor uma boa impressão
%inicial.

%Este texto inicial da dissertação é escrito no fim e resume numa
%página, sem referências externas, o tema e o contexto do trabalho, a
%motivação e os objectivos, as metodologias e técnicas empregues, os
%principais resultados alcançados e as conclusões.

%Este documento ilustra o formato a usar em dissertações na \Feup.
%São dados exemplos de margens, cabeçalhos, títulos, paginação, estilos
%de índices, etc.
%São ainda dados exemplos de formatação de citações, figuras e tabelas,
%equações, referências cruzadas, lista de referências e índices.
%Este documento não pretende exemplificar conteúdos a usar.
%É usado texto descartável, \emph{Loren Ipsum}, para preencher a
%dissertação por forma a ilustrar os formatos.

%Seguem-se umas notas breves mas muito importantes sobre a versão
%provisória e a versão final do documento.
%A versão provisória, depois de verificada pelo orientador e de
%corrigida em contexto pelo autor, deve ser publicada na página
%pessoal de cada estudante/dissertação, juntamente com os dois
%resumos, em português e em inglês; deve manter a marca da água,
%assim como a numeração de linhas conforme aqui se demonstra.

%A versão definitiva, a produzir somente após a defesa, em versão
%impressa (dois exemplares com capas próprias FEUP) e em versão
%eletrónica (6 CDs com "rodela" própria FEUP), deve ser limpa da marca de
%água e da numeração de linhas e deve conter a identificação, na primeira
%página, dos elementos do júri respetivo.
%Deve ainda, se for o caso, ser corrigida de acordo com as instruções
%recebidas dos elementos júri.
