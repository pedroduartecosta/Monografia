\chapter{Conclusions and Future Work} \label{chap:concl}

\section*{}

This chapter reflects on the entire work of this dissertation. Looking at the contributions of the work performed, difficulties and possible paths for improvement and work that can still be done.

\section{Difficulties}
One of the main difficulties around this topic is the complexity of the underlying system, the blockchain. In order to first understand the purpose of oracles, it is required to understand in depth the underlying blockchain. Each blockchain presents a different paradigm, programming language and challenges, and although the oracle problem usually transverses all of them or at least the most used ones, it still requires some adaptation. For this specific work, the author chose the Ethereum for its huge adoption and smart contract features but others, such as the EOS blockchain could have been used.

The other main difficulty is around the oracle topic itself. There's not a lot of documentation around it, especially in the blockchains documentation pages. The only information found is on a few papers and two companies which seem to dominate the market of oracles-as-a-service and are not so transparent in the actual implementation or proof verification.

In the end, in terms of oracle implementation, the author learned from the combined knowledge of a few articles from independent developers who tried to share their knowledge on how they addressed their oracle implementation.


\section{Contributions}
This dissertation intends to pave the way for the development of secure oracle architectures and implementations.

It starts with a \textbf{deep analysis of authenticity proofs}, demystifying its inner working and exposing their limitations as well as scenarios of use so that future developers can be more aware of their usage by existing third-party oracle providers.

Furthermore, it presents a \textbf{trust-oriented guide to architect oracles} backed up by the aforementioned analysis. Exposing the weaknesses and strengths of each architecture, addressing the as-a-service ones regarding their use of proofs. Also exposes the points of trust in a oracle service and how can each one be covered and the trade-offs of doing so as well as relating to existing known uses, with exception for the last architecture.

Finally, to address the lack of known uses in the last architecture, the thesis provides an \textbf{novel implementation of a multi-party self-hosted oracle}. This boilerplate can easily be used and even worked upon and tailor to the needs of distinct smart contract. The author even adds in this implementation the necessary logic for having the oracle being used and deployed by multiple competing parties allowing them to trust in the oracle execution even if they don't trust the other involved parties. This implementation, however, is directed for the Ethereum blockchain, but the author believes that its logic can be ported to other blockchains effortlessly. It also addresses the lack of documentation around building a self-hosted oracle.


\vspace{1cm}
Summing up, the author contributes with a:

\begin{itemize}
    \item Deep analysis of authenticity proofs;
    \item Trust-oriented guide to architect oracles;
    \item Novel implementation of a multi-party self-hosted oracle.
\end{itemize}

The author, also extracted the performed literature review into a paper format, attached in Appendix ,ready to be submitted once a suitable conference is found.

\section{Future Work}

In terms of future work, it would be interesting to analyse use cases of the presented architectures. Not only of oracle services, but use cases in the industry that were created by the same need but not divulged to the community and may fit in the described scenarios. As of now, the author could not find examples to fit all of them and therefore, the author was not confident enough to call them patterns~\citet{Alexander,Gamma1995}. This work would also help in the validation of the assertions made about each architecture.

In terms of the implementation of a self-hosted oracle, the author would like to see some work done in terms of designing a self-hosted that could be applied to a community problem. This means, to a scenario where the stakeholders of the oracle execution are not predefined and can enter or leave the network, more similar to the way blockchain works. Due to time limitations, the author could not think of a feasible way of achieving this without using some kind of proof-of-work, similarly to the blockchain.

% FIXME: Check if this limitation is described in the implementation chapter

\section{Conclusions}

All in all, this dissertation allowed the author to gain a broader knowledge of cryptography, consensus mechanisms and blockchain technology. The author also truly believes that this dissertation is paving the way for future applications of smart-contracts and blockchain technology. Additionally, will help developers who may be struggling using oracles and, unnecessarily, will recur to third-party providers having to support extra costs in their product.