\chapter{Trustable Oracles }\label{chap:chap3}

\section*{}
%Este capítulo deve começar por fazer uma apresentação detalhada do
%problema a resolver\footnote{Na introdução a apresentação do
%  problema foi breve.} podendo mesmo, caso se justifique,
%constituir-se um capítulo com essa finalidade.

%Deve depois dedicar-se à apresentação da solução sem detalhes de
%implementação. 
%Dependendo do trabalho, pode ser uma descrição mais teórica, mais
%``arquitetural'', etc.


\section{Defining Trust}

At this point, a definition of what trust in a oracle is seems appropriate. Trust has a lot of meanings, depending on the needs of all the parties involved. I will model several levels of trust and the requirements and fallacies of each model as well as its application and drawbacks.

Starting from an absolute trust scenario, in this model, the end user, being the smart contract which receives information provided by the oracle, has complete assurances from both the veracity of the data provided by the data-source, as well as, undeniable proof that the oracle did not tamper with the relayed information. This scenario points out two main points of failure, either maliciously or unintentionally. 

The first component which can be faulty or compromised is the data-source. Assuring that the information provided is correct does not have a straightforward answer. What correct means is open to interpretation. For example, if the data source is an IoT sensor, which is prone to failures, being correct is relative. The sensor needs to be perfectly calibrated and accurate. In this case, using several sensors and averaging its values or removing outliers would solve its correctness. Another example, could be an API that returns the current value of the EUR in USD. In this scenario a party that would benefit from a higher conversion than the real one could coerce or attack the data-source into providing a favorable value. The answer here can also be using several data-sources. Another solution would be to use a highly trusted entity such as the European Central Bank (ECB) which can be a lot harder to coerce or attack and having a signatures from the ECB that backs the provided data. Choosing what type of data-source to use has a huge impact on the trust fullness of the provided data not to mention architecture centralization when using a source such as the ECB. All in all, the end user will have to understand the requirements and level of trust necessary.

The second, and most relevant for analysis, is the oracle service used. Oracles are a necessary part of the process, since the other option would be having the data providers adapting to the blockchain which does not seem to be a realistic option at the moment. Therefore we most trust an oracle or a group of oracles. Two main options are available, either trusting a third-party oracle or self deploying an oracle. In the first scenario, three variables take part in the level of trust. Firstly he third-party oracle, if paid for, has the monetary incentive to be honest, since a bad record of dishonesty would have the service loosing credibility and therefore clients. Secondly, by using proofs the oracle can establish its legitimacy, as long as, the proofs can undoubtedly be trusted and verifiable by the smart contract, I will later analyse in depth this issue. Finally, oracle execution transparency by using open-source code and having means for being audit. Additionally to guaranteeing single oracle integrity, it may be in the interest of the user to use several oracles either to provide service availability or to increase trust by combining the result from different oracle services. 


\section{Oracle Architectures}

\subsection{Oracle as a Service w/ Single Data Feed.}

\subsubsection{Context}
Smart contracts will provably power a decentralized world of automation and trust-less commitments. Companies, groups and individuals will be able to automate tasks and contracts but as far as the ecosystem is, at the moment, smart contracts are limited to the information available in the blockchain. Therefore, connecting with the outside world requires a trusted authority to input in the blockchain the required information upon request from the smart contract. This trusted authority is generally called an oracle.

\subsubsection{Example}



\subsubsection{Problem}
As explained before, the deterministic nature of blockchain does not allow smart contracts to directly query a data-feed for information. In this context, oracles surge to empower business and smart contract capabilities, connecting smart contracts to the world outside of the blockchain. The problem here is to trust in the oracle service to not behave maliciously and undermine the trust provided by the blockchain consensus mechanisms. Blockchain technology can be trusted to behave correctly even in byzantine environments, but the oracle service does not abide by the same mechanisms and therefore must provide some kind of proof of honesty. 

\subsubsection{Forces}
\begin{itemize}
\item \textbf{Smart contract empowerment} - Providing smart contracts with trustable information from outside of the blockchain is decisive to gain general adoption and practicality.
\item \textbf{Blockchain Interoperability} - The ability to get information from other blockchains if possible using an oracle that queries one blockchain and inputs the information on another.
\item \textbf{Keeping trust standards} - As blockchain technology creates a trust-less environment, oracles should as well keep up with the level of trust in their functioning.
\end{itemize}


\subsubsection{Solution}
A proposed and existing solution in the market is Oracles as a Service. Oraclize  \cite{Oraclize.it2018OraclizeDocumentation} currently provides fee-based oracles in which smart contracts pay for each request. In order to achieve a higher level of trust, for an extra fee, it can also provide proofs that the provided data as not been tampered by their oracle. The extent to which this proofs can be trusted will be further analysed.

\begin{figure*}[t]
  \begin{center}
    \leavevmode
    \includegraphics[width=0.5\textwidth]{figures/oraclearch1.jpg}
    \caption{Oracle as a Service w/ Single Data Feed.}
    \label{fig:/figures/paper-screening}
  \end{center}
\end{figure*}

\subsubsection{Example Resolved}


\subsubsection{Resulting Context}
This solution results in an architecture that compromises two points of trust. The first being the data-feed itself. No guarantees are given that the data provided is reliable and the smart contract owner must therefore, to the best of his knowledge, select a data-feed in which, by the operator size or record of good behaviour, he can trust.

The second point of failure is the oracle service itself. Although smart contracts, in the resulting context, have access to the information from the outside, that is only possible due to the use of a third party to honestly relay the data. In this architecture, if the oracle simply relays the data, then no trust model can be achieved as the oracle good behavior is not tested against. As this would not be a feasible architecture the existing services provide autenticity proofs to guarantee, to a certain level, their honest behavior. The problem here is on how are these proofs generated, can they be verified on-chain or only off-chain and who is making, or providing, the verification tools. In chapter X I'll deep dive on these questions and technicques. Another reason to trust in the service can be the monetary incentive for good behavior. By paying the oracle for each request, that becomes the oracle service business model and therefore an extensive record of good behavior if crucial for business propesrity and therefore a good enough incentive for honestly conveying the requested data.
In this context, if the autenticity proofs provide enough assurances for the smart contract creator and he trustes in the selected data-feed to provide the required data, then this model can satisfy its needs in terms of trust, as well as, performance, since it only queries one data-feed and uses only one oracle. By not having any consensus mechanism an exchangin the least amount of messages it can both achieve greater performance and a lower cost. But this lower cost and higher performance architecture by itself  is prone to failure due to lack of decentralization and does not guarantee service availability which could lead to a failure in the smart contract to obtain the requested information.




\subsubsection{Known Uses}

\subsection{Oracle as a Service w/ Multiple Data Feeds.}

\subsubsection{Context}
\subsubsection{Example}
\subsubsection{Problem}
\subsubsection{Forces}
\subsubsection{Solution}
\begin{figure*}[t]
  \begin{center}
    \leavevmode
    \includegraphics[width=0.5\textwidth]{figures/oraclearch2.jpg}
    \caption{Oracle as a Service w/ Multiple Data Feeds.}
    \label{fig:/figures/paper-screening}
  \end{center}
\end{figure*}

\subsubsection{Example Resolved}
\subsubsection{Resulting Context}
\subsubsection{Known Uses}

\subsection{Single-Party Self Hosted Oracle.}

\subsubsection{Context}
\subsubsection{Example}
\subsubsection{Problem}
\subsubsection{Forces}
\subsubsection{Solution}

\begin{figure*}[t]
  \begin{center}
    \leavevmode
    \includegraphics[width=0.5\textwidth]{figures/oraclearch3.jpg}
    \caption{Single-Party Self Hosted Oracle.}
    \label{fig:/figures/paper-screening}
  \end{center}
\end{figure*}


\subsubsection{Example Resolved}
\subsubsection{Resulting Context}
\subsubsection{Known Uses}

\subsection{Multi-Party Self Hosted Oracle.}

\subsubsection{Context}
\subsubsection{Example}
\subsubsection{Problem}
\subsubsection{Forces}
\subsubsection{Solution}

\begin{figure*}[t]
  \begin{center}
    \leavevmode
    \includegraphics[width=0.5\textwidth]{figures/oraclearch4.jpg}
    \caption{Multi-Party Self Hosted Oracle.}
    \label{fig:/figures/paper-screening}
  \end{center}
\end{figure*}

\subsubsection{Example Resolved}
\subsubsection{Resulting Context}
\subsubsection{Known Uses}





\section{Summary and conclusions}
